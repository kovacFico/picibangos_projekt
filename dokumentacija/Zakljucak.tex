\chapter{Zaključak i budući rad}

		Zadatak nam je bio napraviti implementaciju kartaške igre u kojoj se igrači bore sa kartama koje mogu skupiti na različitim lokacijama i prijavljuju nove lokacije. Kartografi mogu odobravati lokacije, a administrator može upravljati svim korisnicima i zadavati im privremena isključenja. Na kraju nismo ipak uspjeli implementirati sve što je bilo planirano i neki dijelovi ne rade kako bi trebali prema specifikaciji.
		\\ \\
		Projekt je započeo odlično sa pisanjem dokumentacije i izradom UML dijagrama. Kad je došlo do implementacije napisanog, nastali su neki problemi. Nismo znali točno podijeliti poslove i trebali smo dosta proučavati kako rade radni okviri kojima smo radili front-end i back-end aplikacije. No uz njih smo naučili puno o tim radnim okvirima i vidjeli kakvi se sve problemi mogu riješiti s njima. Za izradu generičkih funkcionalnosti nije bilo prevelikih problema.
		\\ \\
		Težak problem koji smo trebali riješiti je borba. Kako spojiti igrače i kako izvesti borbu i prolazak kroz stanja borbe. Trebalo nam je dosta vremena da ju implementiramo te nije implementirana na najelegantniji način. Ostatak funkcionalnosti išao je lakše. Bilo je malo problema sa implementacijom karte, ali uspjeli smo je ostvariti. Sa funkcionalnostima koje uključuju jednostavan dohvat podataka nije bilo problema.
		\\ \\
		Jedino sa čime smo se dosta mučili bez da smo našli rješenje je spremanje i prikaz slika. Osim što slike nisu implementirane, nismo stigli implementirati ni obrasce uporabe za administratora, i nije implementirano skupljanje lokacija tako da se fizički dođe do njih. Igrači karte mogu skupljati samo iz paketića (maknuli smo ograničenje da to budu samo životinje). Obrasci uporabe koji nisu implementirani: UC3, UC9, UC19, UC25, UC27, UC28, UC29, UC30.
		\\ \\
		Nismo stigli implementirati dosta velik dio aplikacije. Bilo je problema s koordiniranjem zadataka i poslova. Jako puno vremena je otišlo na upoznavanje s alatima. No sada znamo bolje baratati s tim radnim okvirima te bismo mogli projekt napraviti brže.

		 \eject
