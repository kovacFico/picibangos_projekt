\chapter{Zaključak i budući rad}

		Ideja projektnog zadatka je bila napraviti web aplikaciju u Pythonu koristeći moderni web framework FastAPI te Open Source biblioteke poput Streamlita i BeeWarea kako bi ih proanalizirali i upoznali se s njima. Aplikaciju smo kreirali koristeći Streamlit, a ostale biblioteke smo detaljno proučili.
		\\ \\
		FastAPI se istaknuo kao odličan framework za sastavljanje temelja aplikacije. Jednostavan je za korištenje, posebno za one koji su već upoznati Pythonom, dizajniran je kako bi pružao brze i efikasne obrade HTTP zahtjeva te uključuje ugrađene mehanizme sigurnosti.
		\\ \\
		BeeWare nije samo jedan alat, već skup alata i biblioteka koji omogućuju pisanje mobilnih, web i stolnih aplikacija koristeći Python. Ističe se time što nudi mogućnost korištenja Pythona na različitim platformama bez potrebe za mijenjanjem koda. Unatoč tomu, Streamlit se sa svojom jednostavnošću, brzinom razvoja i većom zajednicom koja mu je posvećena jasno ističe kao bolji odabir za rad na web aplikaciji.
		\\ \\
		Streamlit se pokazao kao iznimno sposoban alat za stvaranje web aplikacija. Praktičan je, pruža razne widgete za lakšu interakciju s podatcima, lako se implementira te ima aktivnu i rastuću zajednicu koja doprinosi njegovom razvoju. Osim toga, Streamlit je jedina biblioteka, od svih koje smo analizirali, prilagođena za stvaranje web aplikacija za podatkovnu znanost i strojno učenje.
		\\ \\
		Radom na aplikaciji otkrile su se i mnogobrojne mane navedenih open source biblioteka. Što su alati jednostavniji za koristiti, to su ograničeniji i nestabilniji. Komunikacija s bazom ili nekim drugim frameworkom se teško postiže. Kompliciraniji elementi i widgeti rijetko funkcioniraju zajedno bez grešaka, zato što su mnogi kreirani od različitih korisnika, pa njihova kvaliteta ovisi o vještini programiranja korisnika koji ih je napravio. Streamlit je pogotovo pogođen tim problemima, no zbog velike zajednice ljudi koja radi na njemu, oni su ograničeni samo na njegove najsloženije dijelove.
		\\ \\
		Uzevši u obzir sve vrline i mane, možemo zaključiti da su open-source biblioteke poput Streamlita i BeeWarea korak u dobrom smjeru za razvoj mobilnih i web aplikacija u Pythonu. Iako imaju svoje mane koje ih sprječavaju da se koriste u kompliciranijim aplikacijama, predstavljaju se kao izrazito sposobni alati za brzo i jednostavno stvaranje aplikacija, vizualiziranje i dijeljenje podataka te prikazivanje modela i rezultata strojnog učenja.
		
		 \eject
