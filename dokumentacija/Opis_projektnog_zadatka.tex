\chapter{Opis projektnog zadatka}

		\noindent\textbf{Uvod}\\
		\\
		Osnova projekta je bilo razviti programsku podršku za \textit{Pičibangoš}, web aplikaciju koja predstavlja online planer i nudi korisnicima opcije ugovaranja sastanaka, stvaranja evenata i timova, i online komunikaciju sa prijateljima pojedinačno ili unutar tima. Također, još jedan od zadataka je bio istražiti mogućnosti koje nudi programski jezik Python sa framework-ovima Streamlit i BeeWare u razvijanju frontend-a web aplikacije i u povezivanju sa backendom, isto tako građenim pomoću pythonovog framework-a FastAPI.\\
		\\


		\noindent \textbf{Ciljevi}
		\begin{packed_item}

			\item Istražiti mogućnosti razvoja programske potpore za web pomoću programskog jezika Python
			\item Istražiti Pythonove framework-e za razvoj web aplikacije
			\item Odabir Pythonovih framework-a
			\item Analiza primjene
			\item Zaključak na temlju korištenih tehnologija
		\end{packed_item}
		\bigskip

		\noindent\textbf{Python}\\
		\\
		Python je jedan od najpopularnijih i najučinkovitijih programskih jezika koji sadrži goleme library-e i framowork-e za gotovo svaku tehničku domenu. Pythonovi framowork-ovi automatiziraju implementaciju mnogih zadataka i daju programerima dobru strukturu za razvoj aplikacija. Svaki framowork dolazi s vlastitom kolekcijom modula ili paketa koji značajno skraćuju vrijeme razvoja.\\
		\pagebreak

		\noindent\textbf{Model razvoja}\\
		\\
		Životni ciklus razvoja aplikacija daje proces kojim se aplikacije razvijaju, a modeli
		razvoja aplikacija daju pristup provedbi tog procesa. Ti modeli opisuju kako se sve faze
		životnog ciklusa procesa razvoja softvera spajaju u softverski projekt. Postoji nekoliko
		uobičajenih popularnih modela razvoja i stotine, ako ne i tisuće, njihovih iteracija. Model
		razvoja softvera opisuje što će se raditi ali ostavlja puno prostora načinu kako će se raditi.
		Potrebno je odabrati odgovarajući model ovisno o veličini projekta. Za male projekte je vodopadni model
		bez dodatne dokumentacije i bez restrikcija iterativnog pristupa dobar izbor, te smo u ovom projektu odlučili taj i koristiti.
		\pagebreak

		\section{Opis aplikacije}

		\bigskip

		\noindent\textbf{Uvod}\\
		\\
		Aplikacija Pičibangoš je osmišljena kao planer gdje korisnik ima opciju ugovarati sastanke i događaje, bilo privatne ili poslovne svrhe, te na taj način voditi evidenciju obaveza i događaja. Također aplikacija nudi opciju i povezivanja, te komuniciranja se sa kolegama, pri čemu korisnik na taj način može vidjeti obaveze kolega s kojima je povezan ili kao prijatelj ili preko tima, te time može lakše upravljati i dogovarati događaje koji se tiču više ljudi. Stoga, cilj aplikacije ja olakšati svakodnevnu \\ komunikaciju i suradnju sa kolegama i prijateljima, te time uštediti vrijeme koje bi se neizbježno izgubilo pri dogovaranju sastanaka kroz raspitivanje o obavezama pojedinaca.  \\
		\\

		\noindent\textbf{Korištenje aplikacije}\\
		\\
		Nakon svake uspješne prijave korisnik dobiva pregled svih svojih obaveza koje ima unutar bilo kojeg tima, kao i obaveza koje ima sa pojedinačnim korisnicima na osobnom kalendaru. Iznad kalendara su mu također ponuđene mogućnosti stvaranja novih događaja, timova ili pronalaženja novih prijatelja.  \\
		\\
		Pri odabiru jednog od timova u koje korisnik uključen, kalendar se prilagođava te preciznije prikazuje obaveze koje korisnik ima unutar tima, kao i obaveze koje imaju njegovi timljani. Na taj način korisnik može preciznije odabrati vrijeme u kojem bi mogao kreirati novi događaj, a kojem bi mogli prisustvovati svi timljani. Također korisnik može predložiti jednog od svojih prijatelja ko novog timljana, što odobrava admin koji je stvorio tim, te na temelju toga se i kalendar ažurira.\\
		\\
		Korisnik pri stvaranju novog događaja daje ime događaja, odabire vrijeme u kojem će se spomenuti događaj odvijati i sudionike koje poziva na događaj, bilo pojedinačno navedene ili kroz navođenje timova u kojima su oni prisutni, a korisnik je isto tako sudionik. \\
		\pagebreak

		Pri stvaranju novih timova, korisnik navodi ime tima, njegov opis te odabire prijatelje sa kojima želi stvoriti navedeni tim.\\
		\\
		Ako korisnik želi dodati novog prijatelja, to radi kroz upisivanje imena korisnika u aplikaciji, te kroz slanje zahtjeva prijateljstva, kojeg druga strana mora potvrditi. Na taj način korisnici postaju prijatelji, te mogu stvarati i gledati zajedničke obaveze, kao i imati uvid u obaveze prijatelja.\\
		\pagebreak



		\eject
