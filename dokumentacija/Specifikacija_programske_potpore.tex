\chapter{Specifikacija programske potpore}
	Python framework je zbirka Python modula koja pruža skup zajedničkih funkcija koje se mogu koristiti kao struktura za izgradnju aplikacija bilo koje vrste.\\
	Framework-ovi su osmišljeni kako bi pojednostavili razvojni proces dajući općenite smjernice o tome kako bismo trebali graditi softver i apstrahirali neke od složenijih ili ponavljajućih zadataka, što stoga omogućuje jedinstvenu i prilagođenu logiku za aplikacije.\\
	\\
	Stoga, nakon istraživanja, odlučili smo koristiti navedene Python framework-e u izgradnji aplikacije:

	\begin{packed_enum}

		\item FastAPI
		\item Streamlit
		\item BeeWare

	\end{packed_enum}

	\bigskip

	\section{FastAPI}

	FastAPI je moderan web framework koji je kreirao Sebastián Ramírez za izgradnju RESTful API-ja u Pythonu. Ubrzo je stekao popularnost među programerima zbog svoje jednostavne upotrebe, brzine koja je usporediva sa NodeJS-om i Go-om, i robusnosti. Temelj se zasniva na Pydantic-u, te omogućuje validaciju, serijalizaciju i deserijalizaciju podataka. Također automatski generira OpenAPI dokumentaciju za API-je izgrađene s njim.\\
	Upravo radi navedenih razloga, FastApi je postao najbolji odabir za izgradnju backend-a aplikacije.

	\bigskip

	\section{Streamlit}

	Streamlit je Python framework otvorenog koda koja može lako pretvoriti projekte iz podatkovne znanosti i strojnog učenja u web aplikacije. To je izvrstan odabir za znanstvenike koji se bave podacima ili inženjere strojnog učenja čiji su temeljni skupovi vještina usmjereni na izradu prototipova, eksperimentiranje s modelima i manipuliranje podacima u Pythonu. Streamlit stoga omogućuje stvaranje aplikacije zapanjujućeg izgleda u nekoliko redaka koda, no unatoč navedenim prednostima tu su naravno poneki nedostatci. Najveći utjecaj je imao manjak fleksibilnosti Stramlit-ovih opcija koje su u nekim slučajevima znale otežati razvoj i najjednostvanijih funkcionalnosti.....FILIPE POSERI SE TU

	\bigskip

	\section{BeeWare}

	Python se pokazao kao vrlo sposoban jezik - pristupačan za početnike, ali moćan u rukama stručnjaka. Projekt BeeWare ima za cilj iskoristiti snagu Pythona kao jezika i koristiti ga kako bi korisnicima svih razina vještina omogućio razvoj aplikacija s izvornim korisničkim sučeljima.

	Krajnji cilj projekta BeeWare: biti u mogućnosti učiniti za mobilni i stolni korisnički softver isto što je Django napravio za web softver - dati u ruke korisnicima skup alata i biblioteka koji im omogućuju razvoj bogata izvorna korisnička sučelja i implementirati ih na svoje uređaje. Ovo uključuje:

	Alati za omogućavanje rada Pythona na različitim uređajima,
	Alati za pakiranje Python projekta tako da se može izvoditi na tim uređajima,
	Knjižnice za pristup izvornim widgetima i mogućnostima uređaja,
	Alati koji pomažu u razvoju, otklanjanju pogrešaka, analizi i implementaciji ovih projekata.

	Cilj je da ovaj set alata bude dovoljno jednostavan za korištenje za potpune pridošlice za korištenje u okruženju sličnom Django Girls; ali dovoljno moćni da se mogu koristiti za pokretanje sljedećeg Instagrama, Pinteresta ili Disqusa.

	No, ne bavimo se samo softverom. Također nam je cilj biti projekt s društvenom sviješću. Cilj nam je razviti i održati raznoliku i inkluzivnu zajednicu i imamo Kodeks ponašanja koji se rigorozno provodi. Također težimo razvoju zdrave i održive zajednice - one koja je svjesna problema mentalnog zdravlja svojih sudionika i koja osigurava resurse za ljude da se počnu angažirati i nastave surađivati sa zajednicom.

	\bigskip

	\section{Programski zahtjevi}
	Neovisno o framework-ovima, aplikacija mora ispunjavati sljedeće programske zahtjeve:

		\begin{packed_item}

			\item Sustav nužno mora podržavati rad više korisnika u stvarnom vremenu
			\item Korisničko sučelje i sustav moraju podržavati hrvatsku abecedu
			\item Sustav treba biti implementiran kao web aplikacija koristeći objektno-orijentirane jezike
			\item Neispravno korištenje korisničkog sučelja ne smije narušiti sustav
			\item Sustav treba imati jasno i intuitivno sučelje
			\item Sustavu se može pristupiti iz javne mreže
			\item Zaporke korisnika ne smiju se zapisivati u bazu u plain text formatu

		\end{packed_item}
