\chapter{Specifikacija programske potpore}
	Python framework je zbirka Python modula koja pruža skup zajedničkih funkcija koje se mogu koristiti kao struktura za izgradnju aplikacija bilo koje vrste.\\
	Framework-ovi su osmišljeni kako bi pojednostavili razvojni proces dajući općenite smjernice o tome kako bismo trebali graditi softver i apstrahirali neke od složenijih ili ponavljajućih zadataka, što stoga omogućuje jedinstvenu i prilagođenu logiku za aplikacije.\\
	\\
	Stoga, nakon istraživanja, odlučili smo koristiti navedene Python framework-e u izgradnji aplikacije:

	\begin{packed_enum}

		\item FastAPI
		\item Streamlit
		\item BeeWare

	\end{packed_enum}

	\bigskip

	\section{FastAPI}

	FastAPI je moderan web framework koji je kreirao Sebastián Ramírez za izgradnju RESTful API-ja u Pythonu. Ubrzo je stekao popularnost među programerima zbog svoje jednostavne upotrebe, brzine koja je usporediva sa NodeJS-om i Go-om, i robusnosti. Temelj se zasniva na Pydantic-u, te omogućuje validaciju, serijalizaciju i deserijalizaciju podataka. Također automatski generira OpenAPI dokumentaciju za API-je izgrađene s njim.\\
	Upravo radi navedenih razloga, FastApi je postao najbolji odabir za izgradnju backend-a aplikacije.

	\bigskip

	\section{Streamlit}

	Streamlit, kao otvoreni Python framework, predstavlja snažan alat koji može transformirati projekte u upečatljive web aplikacije. Ovaj okvir posebno privlači znanstvenike koji se bave podacima i inženjere strojnog učenja, pružajući im mogućnost da svoje radove brzo i jednostavno prenesu na web platformu.
	\\ \\
	Streamlit se ističe svojom sposobnošću stvaranja atraktivnih web aplikacija s minimalnim naporom. U nekoliko redaka koda, korisnici mogu transformirati svoje analize i modele u sučelja koja su intuitivna i lako razumljiva. Ova karakteristika čini Streamlit iznimno korisnim za brze iteracije, testiranje ideja i dijeljenje rezultata sa širom publikom.
	\\ \\
	Također, raznolikost widgeta i grafičkih elemenata unutar Streamlita omogućuje korisnicima da prilagode svoje aplikacije prema specifičnim potrebama, čineći ih pristupačnima čak i početnicima u Pythonu. Aktivna zajednica Streamlit korisnika pruža bogatstvo resursa, od korisničkih vodiča do gotovih rješenja, čime olakšava učenje i implementaciju novih funkcionalnosti. Sve ove prednosti čine Streamlit ne samo alatom za brzi razvoj, već i resursom koji promiče laku prilagodbu i inovaciju u svijetu web razvoja.
	\\ \\
	Unatoč tim prednostima, Streamlit nije uvijek najfleksibilniji alat. Njegova snaga leži u jednostavnosti i brzini razvoja, ali to može rezultirati ograničenjima za složenije scenarije ili specifične zahtjeve. Programeri koji traže dublju prilagodbu u dizajnu aplikacija brzo će naići na ograničenja koja neće moći zaobići.
	\\ \\
	Dodatno, jedan od izazova je prilagodba pojedinih komponenti i njihovih međusobnih interakcija. Kada se koriste složenije komponente može doći do sukoba i neusklađenosti. Različiti widgeti ili interaktivni elementi stvoreni od strane različitih korisnika mogu imati različite zahtjeve i uvjete, što rezultira kompatibilnošću koja nije uvijek zajamčena.
	\\ \\
	Usorkos navedenim izazovima, Streamlit se izdvaja kao iznimno koristan alat za brzi razvoj web aplikacija u Pythonu. Njegova jednostavnost i brzina omogućuju korisnicima transformaciju analiza i modela u privlačna korisnička sučelja s minimalnim naporom, dok bi zahtjevi za visokom prilagodljivošću i složenim interakcijama možda zahtijevali razmatranje drugih alata. Stoga je ključno pažljivo odabrati Streamlit kao alat, uzimajući u obzir karakteristike projekta.

	\bigskip

	\section{BeeWare}

	Python, kao programski jezik, ostvario je značajan uspjeh zbog svoje pristupačnosti za početnike, ali i zbog moćnih mogućnosti koje pruža iskusnim programerima. U tom kontekstu, projekt BeeWare izdvaja se kao ambiciozan pokušaj iskorištavanja snage Pythona kako bi omogućio razvoj aplikacija s izvornim korisničkim sučeljima, prilagođenih korisnicima svih razina vještina.
	\\ \\
	Jedinstvenost BeeWare projekta ogleda se u njegovom sveobuhvatnom pristupu razvoju mobilnih i web aplikacija. Cilj mu je stvoriti ekosustav alata i biblioteka koji će omogućiti korisnicima korištenje Pythona na različitim platformama bez složenih prilagodbi koda. Ovaj projekt ističe se u nekoliko ključnih aspekata.
	\\ \\
	Prvo, BeeWare pruža alate za univerzalnost, omogućavajući rad Pythona na različitim uređajima, što širi spektar primjena od mobilnih do stolnih uređaja. Osim toga, projekt nudi alate za pakiranje Python projekata, pojednostavljujući proces izvođenja na različitim uređajima bez potrebe za kompleksnim postupcima prilagodbe.
	\\ \\
	Važan aspekt BeeWare projekta je i pristup izvornim widgetima. Razvoj aplikacija s izvornim korisničkim sučeljem zahtijeva pristup specifičnim widgetima i mogućnostima uređaja, a BeeWare pruža odgovarajuće knjižnice koje olakšavaju ovaj proces.
	\\ \\
	Nadalje, BeeWare se posvećuje podršci za razvoj i analizu projekata. Osim alata za razvoj, projekt se fokusira na otklanjanje pogrešaka, analizu i implementaciju kako bi korisnicima pružio cjelovito iskustvo u razvoju aplikacija.
	\\ \\
	BeeWare ne samo da predstavlja značajan doprinos Python ekosustavu, već i raste u popularnosti unutar Python zajednice. Njegova svestranost i cilj omogućavanja jednostavnog razvoja aplikacija na različitim platformama čine ga obećavajućim projektom koji ima potencijal utjecati na način na koji razmišljamo o mobilnom i stolnom korisničkom softveru. Kroz suradnju i podršku zajednice, BeeWare gradi most između Pythona i izvornih korisničkih sučelja, postavljajući temelje za budućnost aplikacijskog razvoja.

	\bigskip

	\section{Programski zahtjevi}
	Neovisno o framework-ovima, aplikacija mora ispunjavati sljedeće programske zahtjeve:

		\begin{packed_item}

			\item Sustav nužno mora podržavati rad više korisnika u stvarnom vremenu
			\item Korisničko sučelje i sustav moraju podržavati hrvatsku abecedu
			\item Sustav treba biti implementiran kao web aplikacija koristeći objektno-orijentirane jezike
			\item Neispravno korištenje korisničkog sučelja ne smije narušiti sustav
			\item Sustav treba imati jasno i intuitivno sučelje
			\item Sustavu se može pristupiti iz javne mreže
			\item Zaporke korisnika ne smiju se zapisivati u bazu u plain text formatu

		\end{packed_item}
