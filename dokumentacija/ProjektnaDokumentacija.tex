%definira klasu dokumenta
\documentclass[12pt]{report}

%prostor izmedu naredbi \documentclass i \begin{document} se zove uvod. U njemu se nalaze naredbe koje se odnose na cijeli dokument

%osnovni LaTex ne može riješiti sve probleme, pa se koriste različiti paketi koji olakšavaju izradu željenog dokumenta
\usepackage[croatian]{babel}
\usepackage{amssymb}
\usepackage{amsmath}
\usepackage{txfonts}
\usepackage{mathdots}
\usepackage{titlesec}
\usepackage{array}
\usepackage{lastpage}
\usepackage{etoolbox}
\usepackage{tabularray}
\usepackage{color, colortbl}
\usepackage{adjustbox}
\usepackage{geometry}
\usepackage[classicReIm]{kpfonts}
\usepackage{hyperref}
\usepackage{fancyhdr}

\usepackage{float}
\usepackage{setspace}
\restylefloat{table}


\patchcmd{\chapter}{\thispagestyle{plain}}{\thispagestyle{fancy}}{}{} %redefiniranje stila stranice u paketu fancyhdr

%oblik naslova poglavlja
\titleformat{\chapter}{\normalfont\huge\bfseries}{\thechapter.}{20pt}{\Huge}
\titlespacing{\chapter}{0pt}{0pt}{40pt}


\linespread{1.3} %razmak između redaka

\geometry{a4paper, left=1in, top=1in,}  %oblik stranice

\hypersetup{ colorlinks, citecolor=black, filecolor=black, linkcolor=black,	urlcolor=black }   %izgled poveznice


%prored smanjen između redaka u nabrajanjima i popisima
\newenvironment{packed_enum}{
	\begin{enumerate}
		\setlength{\itemsep}{0pt}
		\setlength{\parskip}{0pt}
		\setlength{\parsep}{0pt}
	}{\end{enumerate}}

\newenvironment{packed_item}{
	\begin{itemize}
		\setlength{\itemsep}{0pt}
		\setlength{\parskip}{0pt}
		\setlength{\parsep}{0pt}
	}{\end{itemize}}




%boja za privatni i udaljeni kljuc u tablicama
\definecolor{LightBlue}{rgb}{0.9,0.9,1}
\definecolor{LightGreen}{rgb}{0.9,1,0.9}

%Promjena teksta za dugačke tablice
\DefTblrTemplate{contfoot-text}{normal}{Nastavljeno na idućoj stranici}
\SetTblrTemplate{contfoot-text}{normal}
\DefTblrTemplate{conthead-text}{normal}{(Nastavljeno)}
\SetTblrTemplate{conthead-text}{normal}
\DefTblrTemplate{middlehead,lasthead}{normal}{Nastavljeno od prethodne stranice}
\SetTblrTemplate{middlehead,lasthead}{normal}

%podesavanje zaglavlja i podnožja

\pagestyle{fancy}
\lhead{Projekt R}
\rhead{$Pičibangoš$}
\cfoot{stranica \thepage/\pageref{LastPage}}
\rfoot{\today}
\renewcommand{\headrulewidth}{0.2pt}
\renewcommand{\footrulewidth}{0.2pt}


\begin{document}



	\begin{titlepage}
		\begin{center}
			\vspace*{\stretch{1.0}} %u kombinaciji s ostalim \vspace naredbama definira razmak između redaka teksta
			\LARGE Projekt R\\
			\large Ak. god. 2023./2024.\\

			\vspace*{\stretch{3.0}}

			\huge $Pičibangoš$ $sastančenje$\\
			\Large Dokumentacija projekta\\

			\vspace*{\stretch{12.0}}
			\normalsize
			Sudionici: \textit{Filip Kovač, Filip Barić}\\


			\vspace*{\stretch{1.0}}
			Datum predaje: \textit{26.1.2024.}\\

			\vspace*{\stretch{4.0}}

			Nastavnik: \textit{izv. prof. dr. sc. Vladimir Čeperić}\\

		\end{center}


	\end{titlepage}


	\tableofcontents
	\chapter{Opis projektnog zadatka}

		\noindent\textbf{Uvod}\\
		\\
		Osnova projekta je bilo razviti programsku podršku za \textit{Pičibangoš}, web aplikaciju koja predstavlja online planer i nudi korisnicima opcije ugovaranja sastanaka, stvaranja evenata i timova, i online komunikaciju sa prijateljima pojedinačno ili unutar tima. Također, još jedan od zadataka je bio istražiti mogućnosti koje nudi programski jezik Python sa framework-ovima Streamlit i BeeWare u razvijanju frontend-a web aplikacije i u povezivanju sa backendom, isto tako građenim pomoću pythonovog framework-a FastAPI.\\
		\\


		\noindent \textbf{Ciljevi}
		\begin{packed_item}

			\item Istražiti mogućnosti razvoja programske potpore za web pomoću programskog jezika Python
			\item Istražiti Pythonove framework-e za razvoj web aplikacije
			\item Odabir Pythonovih framework-a
			\item Analiza primjene
			\item Zaključak na temlju korištenih tehnologija
		\end{packed_item}
		\bigskip

		\noindent\textbf{Python}\\
		\\
		Python je jedan od najpopularnijih i najučinkovitijih programskih jezika koji sadrži goleme library-e i framowork-e za gotovo svaku tehničku domenu. Pythonovi framowork-ovi automatiziraju implementaciju mnogih zadataka i daju programerima dobru strukturu za razvoj aplikacija. Svaki framowork dolazi s vlastitom kolekcijom modula ili paketa koji značajno skraćuju vrijeme razvoja.\\
		\pagebreak

		\noindent\textbf{Model razvoja}\\
		\\
		Životni ciklus razvoja aplikacija daje proces kojim se aplikacije razvijaju, a modeli
		razvoja aplikacija daju pristup provedbi tog procesa. Ti modeli opisuju kako se sve faze
		životnog ciklusa procesa razvoja softvera spajaju u softverski projekt. Postoji nekoliko
		uobičajenih popularnih modela razvoja i stotine, ako ne i tisuće, njihovih iteracija. Model
		razvoja softvera opisuje što će se raditi ali ostavlja puno prostora načinu kako će se raditi.
		Potrebno je odabrati odgovarajući model ovisno o veličini projekta. Za male projekte je vodopadni model
		bez dodatne dokumentacije i bez restrikcija iterativnog pristupa dobar izbor, te smo u ovom projektu odlučili taj i koristiti.
		\pagebreak

		\section{Opis aplikacije}

		\bigskip

		\noindent\textbf{Uvod}\\
		\\
		Aplikacija Pičibangoš je osmišljena kao planer gdje korisnik ima opciju ugovarati sastanke i događaje, bilo privatne ili poslovne svrhe, te na taj način voditi evidenciju obaveza i događaja. Također aplikacija nudi opciju i povezivanja, te komuniciranja se sa kolegama, pri čemu korisnik na taj način može vidjeti obaveze kolega s kojima je povezan ili kao prijatelj ili preko tima, te time može lakše upravljati i dogovarati događaje koji se tiču više ljudi. Stoga, cilj aplikacije ja olakšati svakodnevnu \\ komunikaciju i suradnju sa kolegama i prijateljima, te time uštediti vrijeme koje bi se neizbježno izgubilo pri dogovaranju sastanaka kroz raspitivanje o obavezama pojedinaca.  \\
		\\

		\noindent\textbf{Korištenje aplikacije}\\
		\\
		Nakon svake uspješne prijave korisnik dobiva pregled svih svojih obaveza koje ima unutar bilo kojeg tima, kao i obaveza koje ima sa pojedinačnim korisnicima na osobnom kalendaru. Iznad kalendara su mu također ponuđene mogućnosti stvaranja novih događaja, timova ili pronalaženja novih prijatelja.  \\
		\\
		Pri odabiru jednog od timova u koje korisnik uključen, kalendar se prilagođava te preciznije prikazuje obaveze koje korisnik ima unutar tima, kao i obaveze koje imaju njegovi timljani. Na taj način korisnik može preciznije odabrati vrijeme u kojem bi mogao kreirati novi događaj, a kojem bi mogli prisustvovati svi timljani. Također korisnik može predložiti jednog od svojih prijatelja ko novog timljana, što odobrava admin koji je stvorio tim, te na temelju toga se i kalendar ažurira.\\
		\\
		Korisnik pri stvaranju novog događaja daje ime događaja, odabire vrijeme u kojem će se spomenuti događaj odvijati i sudionike koje poziva na događaj, bilo pojedinačno navedene ili kroz navođenje timova u kojima su oni prisutni, a korisnik je isto tako sudionik. \\
		\\
		Pri stvaranju novih timova, korisnik navodi ime tima, njegov opis te odabire prijatelje sa kojima želi stvoriti navedeni tim.\\
		\\
		Ako korisnik želi dodati novog prijatelja, to radi kroz upisivanje imena korisnika u aplikaciji, te kroz slanje zahtjeva prijateljstva, kojeg druga strana mora potvrditi. Na taj način korisnici postaju prijatelji, te mogu stvarati i gledati zajedničke obaveze, kao i imati uvid u obaveze prijatelja.\\
		\pagebreak



		\eject

	\chapter{Specifikacija programske potpore}
	Python framework je zbirka Python modula koja pruža skup zajedničkih funkcija koje se mogu koristiti kao struktura za izgradnju aplikacija bilo koje vrste.\\
	Framework-ovi su osmišljeni kako bi pojednostavili razvojni proces dajući općenite smjernice o tome kako bismo trebali graditi softver i apstrahirali neke od složenijih ili ponavljajućih zadataka, što stoga omogućuje jedinstvenu i prilagođenu logiku za aplikacije.\\
	\\
	Stoga, nakon istraživanja, odlučili smo koristiti navedene Python framework-e u izgradnji aplikacije:

	\begin{packed_enum}

		\item FastAPI
		\item Streamlit
		\item BeeWare

	\end{packed_enum}

	\bigskip

	\section{FastAPI}

	FastAPI je moderan web framework koji je kreirao Sebastián Ramírez za izgradnju RESTful API-ja u Pythonu. Ubrzo je stekao popularnost među programerima zbog svoje jednostavne upotrebe, brzine koja je usporediva sa NodeJS-om i Go-om, i robusnosti. Temelj se zasniva na Pydantic-u, te omogućuje validaciju, serijalizaciju i deserijalizaciju podataka. Također automatski generira OpenAPI dokumentaciju za API-je izgrađene s njim.\\
	Upravo radi navedenih razloga, FastApi je postao najbolji odabir za izgradnju backend-a aplikacije.

	\bigskip

	\section{Streamlit}

	Streamlit je Python framework otvorenog koda koja može lako pretvoriti projekte iz podatkovne znanosti i strojnog učenja u web aplikacije. To je izvrstan odabir za znanstvenike koji se bave podacima ili inženjere strojnog učenja čiji su temeljni skupovi vještina usmjereni na izradu prototipova, eksperimentiranje s modelima i manipuliranje podacima u Pythonu. Streamlit stoga omogućuje stvaranje aplikacije zapanjujućeg izgleda u nekoliko redaka koda, no unatoč navedenim prednostima tu su naravno poneki nedostatci. Najveći utjecaj je imao manjak fleksibilnosti Stramlit-ovih opcija koje su u nekim slučajevima znale otežati razvoj i najjednostvanijih funkcionalnosti.....FILIPE POSERI SE TU

	\bigskip

	\section{BeeWare}

	Python se pokazao kao vrlo sposoban jezik - pristupačan za početnike, ali moćan u rukama stručnjaka. Projekt BeeWare ima za cilj iskoristiti snagu Pythona kao jezika i koristiti ga kako bi korisnicima svih razina vještina omogućio razvoj aplikacija s izvornim korisničkim sučeljima.

	Krajnji cilj projekta BeeWare: biti u mogućnosti učiniti za mobilni i stolni korisnički softver isto što je Django napravio za web softver - dati u ruke korisnicima skup alata i biblioteka koji im omogućuju razvoj bogata izvorna korisnička sučelja i implementirati ih na svoje uređaje. Ovo uključuje:

	Alati za omogućavanje rada Pythona na različitim uređajima,
	Alati za pakiranje Python projekta tako da se može izvoditi na tim uređajima,
	Knjižnice za pristup izvornim widgetima i mogućnostima uređaja,
	Alati koji pomažu u razvoju, otklanjanju pogrešaka, analizi i implementaciji ovih projekata.

	Cilj je da ovaj set alata bude dovoljno jednostavan za korištenje za potpune pridošlice za korištenje u okruženju sličnom Django Girls; ali dovoljno moćni da se mogu koristiti za pokretanje sljedećeg Instagrama, Pinteresta ili Disqusa.

	No, ne bavimo se samo softverom. Također nam je cilj biti projekt s društvenom sviješću. Cilj nam je razviti i održati raznoliku i inkluzivnu zajednicu i imamo Kodeks ponašanja koji se rigorozno provodi. Također težimo razvoju zdrave i održive zajednice - one koja je svjesna problema mentalnog zdravlja svojih sudionika i koja osigurava resurse za ljude da se počnu angažirati i nastave surađivati sa zajednicom.

	\bigskip

	\section{Programski zahtjevi}
	Neovisno o framework-ovima, aplikacija mora ispunjavati sljedeće programske zahtjeve:

		\begin{packed_item}

			\item Sustav nužno mora podržavati rad više korisnika u stvarnom vremenu
			\item Korisničko sučelje i sustav moraju podržavati hrvatsku abecedu
			\item Sustav treba biti implementiran kao web aplikacija koristeći objektno-orijentirane jezike
			\item Neispravno korištenje korisničkog sučelja ne smije narušiti sustav
			\item Sustav treba imati jasno i intuitivno sučelje
			\item Sustavu se može pristupiti iz javne mreže
			\item Zaporke korisnika ne smiju se zapisivati u bazu u plain text formatu

		\end{packed_item}

	\chapter{Zaključak i budući rad}

		Ideja projektnog zadatka je bila napraviti web aplikaciju u Pythonu koristeći moderni web framework FastAPI te Open Source biblioteke poput Streamlita i BeeWarea kako bi ih proanalizirali i upoznali se s njima. Aplikaciju smo kreirali koristeći Streamlit, a ostale biblioteke smo detaljno proučili.
		\\ \\
		FastAPI se istaknuo kao odličan framework za sastavljanje temelja aplikacije. Jednostavan je za korištenje, posebno za one koji su već upoznati Pythonom, dizajniran je kako bi pružao brze i efikasne obrade HTTP zahtjeva te uključuje ugrađene mehanizme sigurnosti.
		\\ \\
		BeeWare nije samo jedan alat, već skup alata i biblioteka koji omogućuju pisanje mobilnih, web i stolnih aplikacija koristeći Python. Ističe se time što nudi mogućnost korištenja Pythona na različitim platformama bez potrebe za mijenjanjem koda. Unatoč tomu, Streamlit se sa svojom jednostavnošću, brzinom razvoja i većom zajednicom koja mu je posvećena jasno ističe kao bolji odabir za rad na web aplikaciji.
		\\ \\
		Streamlit se pokazao kao iznimno sposoban alat za stvaranje web aplikacija. Praktičan je, pruža razne widgete za lakšu interakciju s podatcima, lako se implementira te ima aktivnu i rastuću zajednicu koja doprinosi njegovom razvoju. Osim toga, Streamlit je jedina biblioteka, od svih koje smo analizirali, prilagođena za stvaranje web aplikacija za podatkovnu znanost i strojno učenje.
		\\ \\
		Radom na aplikaciji otkrile su se i mnogobrojne mane navedenih open source biblioteka. Što su alati jednostavniji za koristiti, to su ograničeniji i nestabilniji. Komunikacija s nekim drugim frameworkom se teško postiže, a kompliciraniji elementi i widgeti rijetko funkcioniraju zajedno bez grešaka. Razlog tomu je što su mnogi kreirani od različitih korisnika, pa njihova kvaliteta ovisi o vještini programiranja korisnika koji ih je napravio. Streamlit je pogotovo pogođen tim problemima, no zbog velike zajednice ljudi koja radi na njemu, oni su ograničeni samo na njegove najsloženije dijelove.
		\\ \\
		Uzevši u obzir sve vrline i mane, možemo zaključiti da su open-source biblioteke poput Streamlita i BeeWarea korak u dobrom smjeru za razvoj mobilnih i web aplikacija u Pythonu. Iako imaju svoje mane koje ih sprječavaju da se koriste u kompliciranijim aplikacijama, predstavljaju se kao izrazito sposobni alati za brzo i jednostavno stvaranje aplikacija, vizualiziranje i dijeljenje podataka te prikazivanje modela i rezultata strojnog učenja.

		 \eject

	\chapter*{Popis literature}
		\addcontentsline{toc}{chapter}{Popis literature}


		\begin{enumerate}


			\item  Programsko inženjerstvo, FER ZEMRIS, \url{http://www.fer.hr/predmet/proinz}

			\item  I. Sommerville, "Software engineering", 8th ed, Addison Wesley, 2007.

			\item  T.C.Lethbridge, R.Langaniere, "Object-Oriented Software Engineering", 2nd ed. McGraw-Hill, 2005.

			\item  I. Marsic, Software engineering book``, Department of Electrical and Computer Engineering, Rutgers University, \url{http://www.ece.rutgers.edu/~marsic/books/SE}

			\item  The Unified Modeling Language, \url{https://www.uml-diagrams.org/}

			\item  Astah Community, \url{http://astah.net/editions/uml-new}

			\item  H. Šimić, CROZ, Tehničko predavanje "Razvoj REST web servisa u razvojnom okruženju Spring Boot"

			\item  M. Katanec, J. Totić, CROZ, Tehničko predavanje "Razvoj frontenda u razvojnom okruženju React"

			\item  F. Ricov, CROZ, Tehničko predavanje "Prezentacija i tehničke upute za puštanje web aplikacije u pogon"

			\item  Portal za besplatne fotografije Pixabay \url{https://pixabay.com/}

			\item  Leaflet, an open-source JavaScript library for mobile-friendly interactive maps \url{https://leafletjs.com/}

		\end{enumerate}



	\eject


\end{document} %naredbe i tekst nakon ove naredbe ne ulaze u izgrađen dokument
